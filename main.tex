%!TEX spellcheck = de
\documentclass[german,notoc,titlestyle=tud,draft]{tudbeamer}%note: when switching to the tud titlestyle, an additional build is needed
	%,titlestyle=white,structurestyle=white

\input{shortcuts} 
\bibliography{bibliography.bib}

\usepackage{tabularx,array,ragged2e}

\lstdefinestyle{latex}{
	morecomment=[l]{\%},
	moredelim=*[s][]{\{}{\}},
	moredelim=*[s][]{\[}{\]},
	moredelim=*[is][\color{emph2_3}]{*e}{*}
}

\begin{document}

\title[TUD Beamer]{Die TUD Beamerklasse}
\subtitle{Hinweise zur Verwendung}
\author{Lukas Baron}
\datecity{Dresden, Feb. 2018}
\affiliation[ET-IT \textbullet{} IfA]{Fakultät Elektrotechnik und Informationstechnik \textbullet{} Institut für Automatisierungstechnik}
\additionallogo{IfA_logo_blau}
\setDesignImage{pics/DesignFrameTest_1}

%	In order to let a (sub-)section and slide appear in the navigator bar, build
% 	document twice
%
% 	\section{} %appears in TOC, Column in Navigator
%	\subsection{} %appears in TOC, Line in Navigator
%	\section*{} %does not appear in TOC nor Navigator
%	\subsection*{} %does not appear in TOC nor Navigator

% 	If necessary, the following command can be used to insert an absolutely
%   positioned box. First parameter is the width. X and Y coordinate go to
%   braces. Height adjusts to content.
%
% 	\begin{textblock*}{11cm}(2.05cm, 2.6cm)
%
% 	\end{textblock}


% 
% Frontmatter 
% 
%%%%%%%%%%%%%%%%%%%%%%%%%%%%%%%%%%%%%%%%%%%%%%%%%%%%%%%%%%%%%%%%%%%%%%%%%%%%%%%%%%%%%%%%%%%%%%%%%%%%%%%%%%%%%%%%%%%%%%%%%%%%%
 
%% inserts the title page and the table of contents
\maketitle

% 
% Content 
%
%%%%%%%%%%%%%%%%%%%%%%%%%%%%%%%%%%%%%%%%%%%%%%%%%%%%%%%%%%%%%%%%%%%%%%%%%%%%%%%%%%%%%%%%%%%%%%%%%%%%%%%%%%%%%%%%%%%%%%%%%%%%%
\disableFrameTitleSectionNum
\begin{frame}
	\frametitle{Die TUD Beamerklasse}
	\framesubtitle{Allgemeines vorab}

	Verwendungshinweise:
	\begin{itemize}
		\item Eine Umsetzung des Corporate Designs der TU Dresden für Latex Beamerpräsentationen
		\item Verwendung nur im Rahmen der Angelegenheiten der Universität
		\item Veröffentlichung unter Verwendung der Logos der Universität oder ihrer Struktureinheiten nur mit Erlaubnis
	\end{itemize}
	Download:
	\begin{itemize}
		\item Im Uninetz verfügbar im \href{%
				https://git.agtele.eats.et.tu-dresden.de/agtele-public/latex/de.tud.et.ifa.latex.ifaslides%
			}{%
				\raisebox{-0.3ex}{%
					\beamergotobutton{AG-Tele GitLab%
						\footnote[frame]{\url{https://git.agtele.eats.et.tu-dresden.de/agtele-public/latex/de.tud.et.ifa.latex.ifaslides}}%
					}%
				}%
			}
	\end{itemize}
	Inhalt:
	\begin{itemize}
		\item Teil 1: Anleitung zur Verwendung der Latex-Klasse 
		\item Teil 2: Beispiele und Testfolien
	\end{itemize}
\end{frame}
\enableFrameTitleSectionNum


\part{TUD Beamerklasse}
\section{Die Klasse einbinden}
\begin{frame}[fragile]
	\frametitle{Einbinden der Klasse}
	\framesubtitle{Ohne Versionierung}

	\begin{itemize}
		\item Die notwendigen Dateien werden in das eigene Latex-Projekt kopiert. Notwendige Dateien:
		\begin{itemize}
		 	\item \texttt{tudbeamer.cls}
		 	\item \texttt{TU\_Logo\_SW.pdf}
		\end{itemize}
		\item Die Latex-Klasse wird dann eingebunden per
		\begin{lstlisting}[gobble=6,style=latex,numbers=none]
			\documentclass{tudbeamer}
		\end{lstlisting}
	\end{itemize}
	\begin{itemize}
		\item[+] Keine zusätzliche Software nötig
		\item[+] Klasse liegt im Hauptprojekt
		\item[-] Updates müssen manuell eingepflegt werden
		\item[-] Beispieldokument nicht vorhanden
	\end{itemize}
\end{frame}
\begin{frame}[fragile]
	\frametitle{Einbinden der Klasse}
	\framesubtitle{Mit Versionierung als Submodule}
	\begin{itemize}
		\item Das Git-Projekt der Latex-Klasse wird als Git Submodule in das Hauptprojekt eingebunden.
		\begin{itemize}
			\item Ein neues Git-Repository anlegen: \texttt{git init}
			\item Hinzufügen des Git-Submodule: \texttt{git submodule add "https://git.agtele.eats.et.tu-dresden.de/agtele-public/latex/ de.tud.et.ifa.latex.ifaslides" tudbeamer}
		\end{itemize}
		\item Die Klasse wird dann eingebunden per
		\begin{lstlisting}[gobble=6,style=latex,numbers=none]
			\documentclass{*e./tudbeamer*/tudbeamer}
		\end{lstlisting}
	\end{itemize}	
	\begin{itemize}
		\item[+] Updates per \texttt{git submodule update}
		\item[+] Klasse liegt in einem Unterordner im Hauptprojekt
		\item[+] Beispieldokument und -quellcode im Submodule verfügbar
		\item[-] Zusatzsoftware Git muss installiert werden
	\end{itemize}
\end{frame}
\begin{frame}[fragile]
	\frametitle{Einbinden der Klasse}
	\framesubtitle{Als separates Projekt}

	\begin{itemize}
		\item Das Git-Projekt der Latex-Klasse wird als separates Git Projekt eingebunden.
		\begin{itemize}
			\item Ein neues separates Git-Repository anlegen: 
				\texttt{git clone "https://git.agtele.eats.et.tu-dresden.de/agtele-public/latex/ de.tud.et.ifa.latex.ifaslides"}, z.B. im Ordner \texttt{tudbeamer}.
		\end{itemize}
		\item Die Klasse wird dann eingebunden per
		\begin{lstlisting}[gobble=6,style=latex,numbers=none]
			\documentclass{*e..*/tudbeamer/tudbeamer} %Klassen-Projektordner liegt neben Hauptprojekt
		\end{lstlisting}
	\end{itemize}	
	\begin{itemize}
		\item[+] Updates per \texttt{git pull} im Klassenprojekt
		\item[+] Klasse in mehreren Projekten gleichzeitig verfügbar
		\item[+] Beispieldokument und -quellcode im Submodule verfügbar
		\item[-] Zusatzsoftware Git muss installiert werden
	\end{itemize}
\end{frame}

\section{Klassenoptionen}
\subsection{Optionen}
\begin{frame}[fragile]
	\frametitle{Klassenoptionen}
	\framesubtitle{Modi}

	Grundsätzlich werden alle Optionen der \href{https://ctan.org/tex-archive/macros/latex/contrib/beamer/doc}{		Beamer-Klasse}\footnote{\href{https://ctan.org/tex-archive/macros/latex/contrib/beamer/doc}{
				\url{https://ctan.org/tex-archive/macros/latex/contrib/beamer/doc}}}	
	unterstützt. Es folgen die wichtigsten:
	\begin{itemize}
		\item \emph{handout}: Für die Druckversion werden einige Farben und Einstellungen geändert.
		\begin{lstlisting}[gobble=6,style=latex,numbers=none]
			\documentclass[handout]{tudbeamer}
		\end{lstlisting}
		\item \emph{draft}: Das Erstellen des Dokuments geht schneller, Grafiken werden nicht eingebunden und die Farbverläufe werden nicht gerendert.
		\begin{lstlisting}[gobble=6,style=latex,numbers=none]
			\documentclass[draft]{tudbeamer}
		\end{lstlisting}
		\item \emph{aspectratio}: Das Standard-Seitenverhältnis von 16:9 kann beliebig geändert werden.
		\begin{lstlisting}[gobble=6,style=latex,numbers=none]
			\documentclass[aspectratio=*e43*]{tudbeamer}
		\end{lstlisting} 
	\end{itemize}
\end{frame}

\begin{frame}
	\frametitle{Klassenoptionen}
	\framesubtitle{Gliederungen (TOCs)}

	\begin{itemize}
		\item 
	\end{itemize}

\end{frame}

\subsection{Sonstige Befehle}
\begin{frame}
	\frametitle{Präambelbefehle}

	\begin{itemize}
		\item 
	\end{itemize}

\end{frame}

\begin{frame}
	\frametitle{Klassenfunktionen}

	\begin{itemize}
		\item 
	\end{itemize}

\end{frame}

\section{Corporate Design}
\subsection{Farben}

\begin{frame}
	\frametitle{CD Farben}
	\framesubtitle{Hauptfarben}

	\begin{itemize}
		\item 
	\end{itemize}

\end{frame}
\begin{frame}
	\frametitle{CD Farben}
	\framesubtitle{Auszeichnungsfarben}

	\begin{itemize}
		\item 
	\end{itemize}

\end{frame}
\begin{frame}
	\frametitle{CD Farben}
	\framesubtitle{Sonstige Farben}

	\begin{itemize}
		\item 
	\end{itemize}

\end{frame}

\section{Präsentationsinhalte}
\subsection{Häufige Elemente}
\begin{frame}
	\frametitle{Grafiken}

	\begin{itemize}
		\item 
	\end{itemize}

\end{frame}

\begin{frame}
	\frametitle{Plots}

	\begin{itemize}
		\item 
	\end{itemize}

\end{frame}

\begin{frame}
	\frametitle{Diagramme}

	\begin{itemize}
		\item 
	\end{itemize}

\end{frame}

\begin{frame}
	\frametitle{Listings}

	\begin{itemize}
		\item 
	\end{itemize}

\end{frame}


\newcolumntype{P}[1]{>{\raggedright\hangindent+10pt\hangafter=1\arraybackslash}p{#1}}
\begin{frame}
	\frametitle{Tabellen}
	\begin{columns}[t]
	   	\hspace*{-.00\textwidth}\begin{column}{1.0\textwidth}
	   		\ \\[-3ex]
			\begin{table}
			\hspace*{-.045\textwidth}\begin{tabularx}{.95\textwidth}{@{} P{.33\textwidth} l P{.13\textwidth} l @{}}
			\textbf{Anwendung} & \textbf{Plattform} & \textbf{Inhalt} & \textbf{Ebene}\\
			Bedienpanel Maschine/""Station & in-field, stationär, (Touch-) Panel & interaktiv & P,F,S\\
			Detailmonitor Maschine/""Station & in-field, stationär, Display & fix & P,S,PF (,MES)\\
			Operator-Station/ Leitrechner \newline \ \ \ Maschine/""Station & in-field, stationär, PC & interaktiv & P, PF, MES\\
			\hline%

			Operator-Station Leitwarte Station-Gesamtanlage & stationär, PC & interaktiv & P,PF,MES \\
			Übersichtsbildschirm Leitwarte Gesamtanlage & stationär, Display & gesteuert\newline kontextbasiert & PF,MES \\

			\hline
	
			Inspektionstablets Maschine/""Station/""Feldgerät & in-field, mobil, Tablet & interaktiv\newline kontextbasiert & P,F,S\\

			\hline

			Schichtstatus-/""Performance-Anzeige Maschine/""Station & in-field, stationär, Display & fix & MES\\
			AR Inspektion/""Präsentation Maschine/""Station/""Gerät & in-field, mobil, AR & kontextbasiert & P,F,S,PF

			\end{tabularx}%
			\vspace{-1.5ex}%
			\hspace*{-.05\textwidth}\caption{UIs in Anlagen und Klassifikation, AT Steuerungsebene: \underline{P}rodukt, \underline{F}eld, \underline{S}teuerun g, \underline{PF}/PLT, \underline{MES}, \underline{ERP}}%
			\end{table}
		\end{column}
	\end{columns}

\end{frame}

\subsection{Animationen}
\begin{frame}
	\frametitle{Animationen}

	\begin{itemize}
		\item 
	\end{itemize}

\end{frame}

\setStructureStyle{design}
\part{Template Test Slides}
\setStructureStyle{tud}
\section[Content and Framebreaks]{Testing Slide Content and Framebreaks}	
\subsection*{Content and Framebreaks}
\begin{frame}[allowframebreaks]
	\frametitle{A default Frame}
	sizes:
	\begin{enumerate}
		\item beamerwidth : \the\paperwidth, beamerheight: \the\paperheight
		\item beamerwidth : \number\paperwidth, beamerheight:\number\paperheight
		\item body: \the\bodyx, \the\bodyy; \the\bodywidth, \the\bodywidth
		\item author: \insertauthor
		\item title: \inserttitle
		\item subtitle: \insertsubtitle
		\item datecity: \insertdatecity
		% \def\tmpcolor{}
		% \item hks41: gray - \extractcolorspecs{hks41}{gray}{\tmpcolor} \tmpcolor - cmyk \extractcolorspecs{hks41}{cmyk}{\tmpcolor} \tmpcolor
		\item fourth enumerated item
	\end{enumerate}
	text 

	after 

	the 

	enum

	asdf

	sadf
	
	asdf
	
	sad
	
	fsad
	
	f

	adgsd
	ga

	aerteatr

	earterz

	etzrethbn

	45wrthfhe

	4w65ehjje6ie67
\end{frame}
\begin{frame}[allowframebreaks]
	\frametitle{A default Frame}
	\framesubtitle{A frame subtitle}
	sizes:
	\begin{enumerate}
		\item beamerwidth : \the\paperwidth, beamerheight: \the\paperheight
		\item beamerwidth : \number\paperwidth, beamerheight:\number\paperheight
		\item body: \the\bodyx, \the\bodyy; \the\bodywidth, \the\bodywidth
		\item author: \insertauthor
		\item title: \inserttitle
		\item subtitle: \insertsubtitle
		\item datecity: \insertdatecity
		% \def\tmpcolor{}
		% \item hks41: gray - \extractcolorspecs{hks41}{gray}{\tmpcolor} \tmpcolor - cmyk \extractcolorspecs{hks41}{cmyk}{\tmpcolor} \tmpcolor
		\item fourth enumerated item
	\end{enumerate}
	text 

	after 

	the 

	enum

	asdf

	sadf\footnote{Test Test Test Test Test Test Test Test Test Test Test Test Test Test Test Test Test Test Test Test Test Test Test Test Test Test Test Test Test Test Test Test Test Test Test Test Test Test Test Test }
	
	asdf
	
	sad
	
	fsad
	
	f\footnote{Test 2 Test 2 Test 2 Test 2 Test 2 Test 2 Test 2 Test 2 Test 2 Test 2 Test 2 Test 2 Test 2 Test 2 Test 2 Test 2 Test 2 Test 2 Test 2 Test 2 Test 2 Test 2 Test 2 Test 2 Test 2 Test 2 Test 2 Test 2 Test 2 Test 2 Test 2 Test 2 Test 2 Test 2 Test 2 Test 2 Test 2 Test 2 Test 2 Test 2 }

	adgsd
	ga

	aerteatr

	earterz

	etzrethbn

	45wrthfhe

	%4w65ehjje6ie67

	% 0
	
	% 1

	% 2
	
	% 3

	% 4

	% 5

	% 6

	% er
	
	% teq
	
	% t

	% ert
\end{frame}
\begin{frame}[t]
	\frametitle{A default Frame top aligned}
	\framesubtitle{A frame subtitle}
	sizes:
	\begin{enumerate}
		\item beamerwidth : \the\paperwidth, beamerheight: \the\paperheight
		\item beamerwidth : \number\paperwidth, beamerheight:\number\paperheight
		\item body: \the\bodyx, \the\bodyy; \the\bodywidth, \the\bodywidth
		\item author: \insertauthor
		\item title: \inserttitle
		\item subtitle: \insertsubtitle
		\item datecity: \insertdatecity
		% \def\tmpcolor{}
		% \item hks41: gray - \extractcolorspecs{hks41}{gray}{\tmpcolor} \tmpcolor - cmyk \extractcolorspecs{hks41}{cmyk}{\tmpcolor} \tmpcolor
		\item fourth enumerated item
	\end{enumerate}
	text after the enum
\end{frame}
\begin{frame}[t]
	repeat top without title:
	\begin{enumerate}
		\item beamerwidth : \the\paperwidth, beamerheight: \the\paperheight
		\item beamerwidth : \number\paperwidth, beamerheight:\number\paperheight
		\item body: \the\bodyx, \the\bodyy; \the\bodywidth, \the\bodywidth
		\item author: \insertauthor
		\item title: \inserttitle
		\item subtitle: \insertsubtitle
		\item datecity: \insertdatecity
		% \def\tmpcolor{}
		% \item hks41: gray - \extractcolorspecs{hks41}{gray}{\tmpcolor} \tmpcolor - cmyk \extractcolorspecs{hks41}{cmyk}{\tmpcolor} \tmpcolor
		\item fourth enumerated item
	\end{enumerate}
	text after the enum
\end{frame}
\begin{frame}
	repeat centered without title:
	\begin{enumerate}
		\item beamerwidth : \the\paperwidth, beamerheight: \the\paperheight
		\item beamerwidth : \number\paperwidth, beamerheight:\number\paperheight
		\item body: \the\bodyx, \the\bodyy; \the\bodywidth, \the\bodywidth
		\item author: \insertauthor
		\item title: \inserttitle
		\item subtitle: \insertsubtitle
		\item datecity: \insertdatecity
		% \def\tmpcolor{}
		% \item hks41: gray - \extractcolorspecs{hks41}{gray}{\tmpcolor} \tmpcolor - cmyk \extractcolorspecs{hks41}{cmyk}{\tmpcolor} \tmpcolor
		\item fourth enumerated item
	\end{enumerate}
	text after the enum
\end{frame}
\begin{frame}[t]
	\frametitle{A slide using columns}
	\begin{columns}
	   	\begin{column}{.5\textwidth}
	   		Some text askfnaksl \textbf{bold test} jfep oruht \emph{emphasized text} jkaiiaeun asdf qep  asdfkj  
	   		\begin{itemize}
	   			\item \textbf{An Item with long text} llk asdklwpk s98776sdfßsß0j nvfo ÄÜöösaödfSADF
	   			\begin{itemize}
	   				\item Sub item
	   				\item more sub items
	   			\end{itemize}
	   		\end{itemize}
		\end{column}
	   	\begin{column}{.5\textwidth}
	   		Some text askfnaksl \textbf{bold test} jfep oruht \emph{emphasized text} jkaiiaeun asdf qep  asdfkj  
	   		\begin{itemize}
	   			\item \textbf{An Item with long text} llk asdklwpk s98776sdfßsß0j nvfo ÄÜöösaödfSADF
	   			\begin{itemize}
	   				\item Sub item
	   				\item more sub items
	   			\end{itemize}
	   		\end{itemize}
		\end{column}
	\end{columns}
\end{frame}

\subsection{Subsection 2}
\begin{frame}
	\frametitle{1st frame of 2nd subsection}
	\framesubtitle{A slide with an animated image using overlays}
	\centering
some content
	% \onslide<1>\includegraphics[width=1\textwidth,trim=.02cm .02cm .02cm .02cm, clip]{pics/animated_0}%
	% \onslide<2>\includegraphics[width=1\textwidth,trim=.02cm .02cm .02cm .02cm, clip]{pics/animated_1}%
cccd
\end{frame}

\begin{frame}
	\frametitle{2nd frame of 2nd subsection}
	\begin{columns}[t]
		\begin{column}{.2\textwidth}
			some explanatory text for the image\\
			TODO adjust column separators
		\end{column}
	   	\begin{column}{.80\textwidth}
	   		\ \\[-2ex]
	   		some content
			%\centering\includegraphics[keepaspectratio=true,width=1\textwidth,trim=.02cm .02cm .02cm .02cm, clip]{pics/image}%
		\end{column}
	\end{columns}
\end{frame}

\section[ShortSection2]{Long Section 2 Title}
\subsection{Subsection 2.1}
\begin{frame}
	\frametitle{A default Frame}
	\framesubtitle{A frame subtitle}

	Normaltext:
	\begin{enumerate}
		\item enumerated item
		\item second enumerated item
		\item third enumerated item
		\begin{enumerate}
			\item enumerated item
			\item second enumerated item
			\item third enumerated item
			\item fourth enumerated item
		\end{enumerate}
		\item fourth enumerated item
	\end{enumerate}
\end{frame}
\begin{frame}
	\frametitle{A default Frame}
	\framesubtitle{A frame subtitle}
	Normaltext:
	\begin{itemize}
		\item item item
		\item second  item
		\item third  item
		\begin{itemize}
			\item  item
			\item second  item
			\item third  item
			\item fourth  item
		\end{itemize}
		\item fourth  item
	\end{itemize}

\end{frame}
\begin{frame}
	\frametitle{A slide using columns}
	\begin{columns}
	   	\begin{column}{.5\textwidth}
	   		Some text askfnaksl \textbf{bold test} jfep oruht \emph{emphasized text} jkaiiaeun asdf qep  asdfkj  
	   		\begin{itemize}
	   			\item \textbf{An Item with long text} llk asdklwpk s98776sdfßsß0j nvfo ÄÜöösaödfSADF
	   			\begin{itemize}
	   				\item Sub item
	   				\item more sub items
	   			\end{itemize}
	   		\end{itemize}
		\end{column}
	   	\begin{column}{.5\textwidth}
	   		Some text askfnaksl \textbf{bold test} jfep oruht \emph{emphasized text} jkaiiaeun asdf qep  asdfkj  
	   		\begin{itemize}
	   			\item \textbf{An Item with long text} llk asdklwpk s98776sdfßsß0j nvfo ÄÜöösaödfSADF
	   			\begin{itemize}
	   				\item Sub item
	   				\item more sub items
	   			\end{itemize}
	   		\end{itemize}
		\end{column}
	\end{columns}
\end{frame}

%
% Backmatter
%
%%%%%%%%%%%%%%%%%%%%%%%%%%%%%%%%%%%%%%%%%%%%%%%%%%%%%%%%%%%%%%%%%%%%%%%%%%%%%%%%%%%%%%%%%%%%%%%%%%%%%%%%%%%%%%%%%%%%%%%%%%%%%

\makeatletter
\disableSectionFrame%
%\disableFrameTitleSectionNum%
\section*{Backmatter}
\begin{finalframe}
	\frametitle*{Test Backmatter Frame}
	\centering{test}

	textwidth: \the\textwidth\\
	margin left: \the\bodyx\\
	%margin right: \the\beamer@rightmargin\\
	body width: \the\bodywidth\\
	paper width: \the\beamer@paperwidth\\

	TODO: der rechte margin stimmt nicht ganz: 27.45247pt

\end{finalframe}
\makeatother

\begin{finalframe}
	\vspace*{2ex}\centering\includegraphics[keepaspectratio=true,width=0.8\textwidth]{br_logo_weiss.png}
\end{finalframe}

%\toclesssectionstar{}
% \regardspage{
% 	\vspace*{1cm}\centering\includegraphics[keepaspectratio=true,width=0.8\textwidth]{br_logo_weiss.png}
% }
% \regardspage{
% 	
% 	}

% %TODO rework the invocation of special pages
% \regardspage{
% 	\vskip4ex\begin{columns}
% 	    \begin{column}{0.5\textwidth}        
% 	        \huge Vielen Dank\\ für Ihre Aufmerksamkeit\\[1ex]\
% 	    \end{column}
	    
% 	    \begin{column}{0.3\textwidth}
% 	        \flushright{\includegraphics[keepaspectratio=true,width=.6\columnwidth]{pics/logo_image}}
%         	\vspace{1ex}\\

% 	        \large My NAme\\[0.5ex]
% 	        \scriptsize
% 	        my.mail@tu-dresden.de\\[4ex]
% 	        Where my office is 42\\        
% 	        Building/Room\\
%         	Dresden, Germany\\[4ex]\

% 	    \end{column}
% 	\end{columns}
% }
% \backmatterpage{
%     \frametitle*{Credits}
%     \begin{enumerate}
%     	\item nothing here yet
%   %       \item "`Pueblo Chemical Agent - Agitated reactor (motor)"' von "`PEO,
% 		%     Assembled Chemical Weapons Alternatives"', Veröffentlicht unter
% 		%     CC-BY 2.0\,--\,Folie ?
		    
% 		% \item "`Interfaz hmi"' von "`Italogio8 (Eigenes Werk)"',
% 		% 	Veröffentlicht unter CC BY-SA 4.0\,--\,Folie ?
		    
% 		% \item "`Manufacturing equipment 110"' von "`Mixabest"', 
% 		% 	Veröffentlicht unter CC BY-SA 3.0\,--\,Folie ?
			
%     \end{enumerate}
% }

% % additional pages for Q&A purposes
\begin{backmatterframe}
	\frametitle*{Feedback}
	\centering\Huge ? 
\end{backmatterframe}

% %reset total frame counter in order to hide pages behind actual content.
% \addtocounter{framenumber}{-2}

\end{document}

  	
