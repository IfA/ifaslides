%!TEX spellcheck = de
\documentclass[german,aspectratio=169,notoc,draft]{tudbeamer}%note: when switching to the tud titlestyle, an additional build is needed
	%,titlestyle=white,structurestyle=white

%TODO cleanup
% \makeatletter
% \newdimen\beamer@miniframeradius
% \beamer@miniframeradius=0.05cm
% \define@key{beamer@margin}{mini frame radius}{\beamer@miniframeradius=#1\relax}
% \setbeamertemplate{mini frame}
% {%
%   \begin{pgfpicture}{0pt}{0pt}{2\beamer@miniframeradius}{2.5\beamer@miniframeradius}
%     \pgfpathcircle{\pgfpoint{\beamer@miniframeradius}{\beamer@miniframeradius}}{\beamer@miniframeradius}
%     \pgfusepath{fill,stroke}
%   \end{pgfpicture}
% }
% \setbeamertemplate{mini frame in current subsection}
% {%
%   \begin{pgfpicture}{0pt}{0pt}{2\beamer@miniframeradius}{2.5\beamer@miniframeradius}
%     \pgfpathcircle{\pgfpoint{\beamer@miniframeradius}{\beamer@miniframeradius}}{\beamer@miniframeradius}
%     \pgfusepath{stroke}
%   \end{pgfpicture}%
% }
% \makeatother

% \setbeamersize{mini frame radius=.04cm,mini frame size=.12cm,mini frame
% offset=-.02cm}
%TODO end

\input{shortcuts} 
\bibliography{bibliography.bib}


\usepackage{tabularx,array,ragged2e}
%\usepackage[overlay,absolute]{textpos}
%\setbeamerfont{caption}{size=\scriptsize}

\begin{document}

% \einrichtung{Fakultät Elektrotechnik und Informationstechnik}
% \institut{$\bullet$ Institut für Automatisierungstechnik}

% \newcommand{\lstretch}{\normalsize\strut \Large\ }

\title[DA Verteidigung]{Verteidigung der Diplomarbeit}
\subtitle{IT\,--\, Chancen und Risiken}
\author{Maxima Mustermann}
\datecity{Dresden, Feb. 2018}
\affiliation[ET-IT \textbullet{} IfA]{Fakultät Elektrotechnik und Informationstechnik \textbullet{} Institut für Automatisierungstechnik}
\additionallogo{IfA_logo_blau}
 
%	In order to let a (sub-)section and slide appear in the navigator bar, build
% 	document twice
%
% 	\section{} %appears in TOC, Column in Navigator
%	\subsection{} %appears in TOC, Line in Navigator
%	\section*{} %does not appear in TOC nor Navigator
%	\subsection*{} %does not appear in TOC nor Navigator

% 	If necessary, the following command can be used to insert an absolutely
%   positioned box. First parameter is the width. X and Y coordinate go to
%   braces. Height adjusts to content.
%
% 	\begin{textblock*}{11cm}(2.05cm, 2.6cm)
%
% 	\end{textblock}


% 
% Frontmatter 
% 
%%%%%%%%%%%%%%%%%%%%%%%%%%%%%%%%%%%%%%%%%%%%%%%%%%%%%%%%%%%%%%%%%%%%%%%%%%%%%%%%%%%%%%%%%%%%%%%%%%%%%%%%%%%%%%%%%%%%%%%%%%%%%
 
%% inserts the title page
	\maketitle

% 
% Content 
%
%%%%%%%%%%%%%%%%%%%%%%%%%%%%%%%%%%%%%%%%%%%%%%%%%%%%%%%%%%%%%%%%%%%%%%%%%%%%%%%%%%%%%%%%%%%%%%%%%%%%%%%%%%%%%%%%%%%%%%%%%%%%%
\part{The main part of the presentation}
\section[ShortSection]{Long Section Title}	
\subsection{Subsection}
\begin{frame}[allowframebreaks]
	\frametitle{A default Frame}
	sizes:
	\begin{enumerate}
		\item beamerwidth : \the\paperwidth, beamerheight: \the\paperheight
		\item beamerwidth : \number\paperwidth, beamerheight:\number\paperheight
		\item body: \the\bodyx, \the\bodyy; \the\bodywidth, \the\bodywidth
		\item author: \insertauthor
		\item title: \inserttitle
		\item subtitle: \insertsubtitle
		\item datecity: \insertdatecity
		% \def\tmpcolor{}
		% \item hks41: gray - \extractcolorspecs{hks41}{gray}{\tmpcolor} \tmpcolor - cmyk \extractcolorspecs{hks41}{cmyk}{\tmpcolor} \tmpcolor
		\item fourth enumerated item
	\end{enumerate}
	text 

	after 

	the 

	enum

	asdf

	sadf
	
	asdf
	
	sad
	
	fsad
	
	f

	adgsd
	ga

	aerteatr

	earterz

	etzrethbn

	45wrthfhe

	4w65ehjje6ie67
\end{frame}
\begin{frame}[allowframebreaks]
	\frametitle{A default Frame}
	\framesubtitle{A frame subtitle}
	sizes:
	\begin{enumerate}
		\item beamerwidth : \the\paperwidth, beamerheight: \the\paperheight
		\item beamerwidth : \number\paperwidth, beamerheight:\number\paperheight
		\item body: \the\bodyx, \the\bodyy; \the\bodywidth, \the\bodywidth
		\item author: \insertauthor
		\item title: \inserttitle
		\item subtitle: \insertsubtitle
		\item datecity: \insertdatecity
		% \def\tmpcolor{}
		% \item hks41: gray - \extractcolorspecs{hks41}{gray}{\tmpcolor} \tmpcolor - cmyk \extractcolorspecs{hks41}{cmyk}{\tmpcolor} \tmpcolor
		\item fourth enumerated item
	\end{enumerate}
	text 

	after 

	the 

	enum

	asdf

	sadf
	
	asdf
	
	sad
	
	fsad
	
	f

	adgsd
	ga

	aerteatr

	earterz

	etzrethbn

	45wrthfhe

	4w65ehjje6ie67

	0
	
	1

	2
	
	3

	4

	5

	6

	er
	
	teq
	
	t

	ert
\end{frame}
\begin{frame}[t]
	\frametitle{A default Frame top aligned}
	\framesubtitle{A frame subtitle}
	sizes:
	\begin{enumerate}
		\item beamerwidth : \the\paperwidth, beamerheight: \the\paperheight
		\item beamerwidth : \number\paperwidth, beamerheight:\number\paperheight
		\item body: \the\bodyx, \the\bodyy; \the\bodywidth, \the\bodywidth
		\item author: \insertauthor
		\item title: \inserttitle
		\item subtitle: \insertsubtitle
		\item datecity: \insertdatecity
		% \def\tmpcolor{}
		% \item hks41: gray - \extractcolorspecs{hks41}{gray}{\tmpcolor} \tmpcolor - cmyk \extractcolorspecs{hks41}{cmyk}{\tmpcolor} \tmpcolor
		\item fourth enumerated item
	\end{enumerate}
	text after the enum
\end{frame}
\begin{frame}[t]
	repeat top without title:
	\begin{enumerate}
		\item beamerwidth : \the\paperwidth, beamerheight: \the\paperheight
		\item beamerwidth : \number\paperwidth, beamerheight:\number\paperheight
		\item body: \the\bodyx, \the\bodyy; \the\bodywidth, \the\bodywidth
		\item author: \insertauthor
		\item title: \inserttitle
		\item subtitle: \insertsubtitle
		\item datecity: \insertdatecity
		% \def\tmpcolor{}
		% \item hks41: gray - \extractcolorspecs{hks41}{gray}{\tmpcolor} \tmpcolor - cmyk \extractcolorspecs{hks41}{cmyk}{\tmpcolor} \tmpcolor
		\item fourth enumerated item
	\end{enumerate}
	text after the enum
\end{frame}
\begin{frame}
	repeat centered without title:
	\begin{enumerate}
		\item beamerwidth : \the\paperwidth, beamerheight: \the\paperheight
		\item beamerwidth : \number\paperwidth, beamerheight:\number\paperheight
		\item body: \the\bodyx, \the\bodyy; \the\bodywidth, \the\bodywidth
		\item author: \insertauthor
		\item title: \inserttitle
		\item subtitle: \insertsubtitle
		\item datecity: \insertdatecity
		% \def\tmpcolor{}
		% \item hks41: gray - \extractcolorspecs{hks41}{gray}{\tmpcolor} \tmpcolor - cmyk \extractcolorspecs{hks41}{cmyk}{\tmpcolor} \tmpcolor
		\item fourth enumerated item
	\end{enumerate}
	text after the enum
\end{frame}
\begin{frame}[t]
	\frametitle{A slide using columns}
	\begin{columns}
	   	\begin{column}{.5\textwidth}
	   		Some text askfnaksl \textbf{bold test} jfep oruht \emph{emphasized text} jkaiiaeun asdf qep  asdfkj  
	   		\begin{itemize}
	   			\item \textbf{An Item with long text} llk asdklwpk s98776sdfßsß0j nvfo ÄÜöösaödfSADF
	   			\begin{itemize}
	   				\item Sub item
	   				\item more sub items
	   			\end{itemize}
	   		\end{itemize}
		\end{column}
	   	\begin{column}{.5\textwidth}
	   		Some text askfnaksl \textbf{bold test} jfep oruht \emph{emphasized text} jkaiiaeun asdf qep  asdfkj  
	   		\begin{itemize}
	   			\item \textbf{An Item with long text} llk asdklwpk s98776sdfßsß0j nvfo ÄÜöösaödfSADF
	   			\begin{itemize}
	   				\item Sub item
	   				\item more sub items
	   			\end{itemize}
	   		\end{itemize}
		\end{column}
	\end{columns}
\end{frame}

\subsection{Subsection 2}
\begin{frame}
	\frametitle{1st frame of 2nd subsection}
	\framesubtitle{A slide with an animated image using overlays}
	\centering
some content
	% \onslide<1>\includegraphics[width=1\textwidth,trim=.02cm .02cm .02cm .02cm, clip]{pics/animated_0}%
	% \onslide<2>\includegraphics[width=1\textwidth,trim=.02cm .02cm .02cm .02cm, clip]{pics/animated_1}%
cccd
\end{frame}

\begin{frame}
	\frametitle{2nd frame of 2nd subsection}
	\begin{columns}[t]
		\begin{column}{.2\textwidth}
			some explanatory text for the image\\
			TODO adjust column separators
		\end{column}
	   	\begin{column}{.80\textwidth}
	   		\ \\[-2ex]
	   		some content
			%\centering\includegraphics[keepaspectratio=true,width=1\textwidth,trim=.02cm .02cm .02cm .02cm, clip]{pics/image}%
		\end{column}
	\end{columns}
\end{frame}

\section[ShortSection2]{Long Section 2 Title}
\subsection{Subsection 2.1}
\begin{frame}
	\frametitle{A default Frame}
	\framesubtitle{A frame subtitle}

	Normaltext:
	\begin{enumerate}
		\item enumerated item
		\item second enumerated item
		\item third enumerated item
		\item fourth enumerated item
	\end{enumerate}
\end{frame}
\begin{frame}
	\frametitle{A slide using columns}
	\begin{columns}
	   	\begin{column}{.5\textwidth}
	   		Some text askfnaksl \textbf{bold test} jfep oruht \emph{emphasized text} jkaiiaeun asdf qep  asdfkj  
	   		\begin{itemize}
	   			\item \textbf{An Item with long text} llk asdklwpk s98776sdfßsß0j nvfo ÄÜöösaödfSADF
	   			\begin{itemize}
	   				\item Sub item
	   				\item more sub items
	   			\end{itemize}
	   		\end{itemize}
		\end{column}
	   	\begin{column}{.5\textwidth}
	   		Some text askfnaksl \textbf{bold test} jfep oruht \emph{emphasized text} jkaiiaeun asdf qep  asdfkj  
	   		\begin{itemize}
	   			\item \textbf{An Item with long text} llk asdklwpk s98776sdfßsß0j nvfo ÄÜöösaödfSADF
	   			\begin{itemize}
	   				\item Sub item
	   				\item more sub items
	   			\end{itemize}
	   		\end{itemize}
		\end{column}
	\end{columns}
\end{frame}

\part{A 2nd Part}
\section{test}
\subsection*{Subsection not in the TOC}
\begin{frame}[allowframebreaks]
	\frametitle{A blanc frame TODO}
	This frame should be clear of footer and header stuff since it is a blanc frame
\end{frame}

\subsection{Subsection in the TOC}
\begin{frame}
	\frametitle{Hey Ho}
	\framesubtitle{Test subtitle}
	Can I have a Formula?
\end{frame}
\begin{frame}
	\frametitle{Hey Ho}
	\framesubtitle{Test subtitle}
	Can I have a Tabular?
\end{frame}

\subsection{Subsection in the TOC}
\begin{frame}
	\frametitle{Hey Ho}
	\framesubtitle{Test subtitle}
	Can I have a Formula?
\end{frame}
\begin{frame}
	\frametitle{Hey Ho}
	\framesubtitle{Test subtitle}
	Can I have a Tabular?
\end{frame}

\newcolumntype{P}[1]{>{\raggedright\hangindent+10pt\hangafter=1\arraybackslash}p{#1}}
\begin{frame}
	\frametitle{UI Engineering 2}
	\framesubtitle{Einsatzmöglichkeiten verschiedener UI Plattformen}

	\begin{columns}[t]
	   	\hspace*{-.00\textwidth}\begin{column}{1.0\textwidth}
	   		\ \\[-5ex]
			\begin{table}
			\hspace*{-.045\textwidth}\begin{tabularx}{.95\textwidth}{@{} P{.33\textwidth} l P{.13\textwidth} l @{}}
			\textbf{Anwendung} & \textbf{Plattform} & \textbf{Inhalt} & \textbf{Ebene}\\
			Bedienpanel Maschine/""Station & in-field, stationär, (Touch-) Panel & interaktiv & P,F,S\\
			Detailmonitor Maschine/""Station & in-field, stationär, Display & fix & P,S,PF (,MES)\\
			Operator-Station/ Leitrechner \newline \ \ \ Maschine/""Station & in-field, stationär, PC & interaktiv & P, PF, MES\\
			\hline%

			Operator-Station Leitwarte Station-Gesamtanlage & stationär, PC & interaktiv & P,PF,MES \\
			Übersichtsbildschirm Leitwarte Gesamtanlage & stationär, Display & gesteuert\newline kontextbasiert & PF,MES \\

			\hline
	
			Inspektionstablets Maschine/""Station/""Feldgerät & in-field, mobil, Tablet & interaktiv\newline kontextbasiert & P,F,S\\

			\hline

			Schichtstatus-/""Performance-Anzeige Maschine/""Station & in-field, stationär, Display & fix & MES\\
			AR Inspektion/""Präsentation Maschine/""Station/""Gerät & in-field, mobil, AR & kontextbasiert & P,F,S,PF

			\end{tabularx}
			\vspace{-1.5ex}
			\hspace*{-.05\textwidth}\caption{UIs in Anlagen und Klassifikation, AT Steuerungsebene: \underline{P}rodukt, \underline{F}eld, \underline{S}teuerung, \underline{PF}/PLT, \underline{MES}, \underline{ERP}}
			\end{table}
		\end{column}
	\end{columns}
\end{frame}

%
% Backmatter
%
%%%%%%%%%%%%%%%%%%%%%%%%%%%%%%%%%%%%%%%%%%%%%%%%%%%%%%%%%%%%%%%%%%%%%%%%%%%%%%%%%%%%%%%%%%%%%%%%%%%%%%%%%%%%%%%%%%%%%%%%%%%%%

%\toclesssectionstar{}
% \regardspage{
% 	\vspace*{1cm}\centering\includegraphics[keepaspectratio=true,width=0.8\textwidth]{br_logo_weiss.png}
% }
% \regardspage{
% 	\vspace*{1cm}\centering\includegraphics[keepaspectratio=true,width=0.8\textwidth]{br_logo_farbig.png}
% 	}

% %TODO rework the invocation of special pages
% \regardspage{
% 	\vskip4ex\begin{columns}
% 	    \begin{column}{0.5\textwidth}        
% 	        \huge Vielen Dank\\ für Ihre Aufmerksamkeit\\[1ex]\
% 	    \end{column}
	    
% 	    \begin{column}{0.3\textwidth}
% 	        \flushright{\includegraphics[keepaspectratio=true,width=.6\columnwidth]{pics/logo_image}}
%         	\vspace{1ex}\\

% 	        \large My NAme\\[0.5ex]
% 	        \scriptsize
% 	        my.mail@tu-dresden.de\\[4ex]
% 	        Where my office is 42\\        
% 	        Building/Room\\
%         	Dresden, Germany\\[4ex]\

% 	    \end{column}
% 	\end{columns}
% }
% \backmatterpage{
%     \frametitle*{Credits}
%     \begin{enumerate}
%     	\item nothing here yet
%   %       \item "`Pueblo Chemical Agent - Agitated reactor (motor)"' von "`PEO,
% 		%     Assembled Chemical Weapons Alternatives"', Veröffentlicht unter
% 		%     CC-BY 2.0\,--\,Folie ?
		    
% 		% \item "`Interfaz hmi"' von "`Italogio8 (Eigenes Werk)"',
% 		% 	Veröffentlicht unter CC BY-SA 4.0\,--\,Folie ?
		    
% 		% \item "`Manufacturing equipment 110"' von "`Mixabest"', 
% 		% 	Veröffentlicht unter CC BY-SA 3.0\,--\,Folie ?
			
%     \end{enumerate}
% }

% % additional pages for Q&A purposes
% \backmatterpage{
% 	\frametitle*{Feedback}
% 	\centering\Huge ? 
% }

% %reset total frame counter in order to hide pages behind actual content.
% \addtocounter{framenumber}{-2}

\end{document}

  	
